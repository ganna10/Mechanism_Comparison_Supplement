
\begin{figure}
    \begin{center}
        \includegraphics[width=\textwidth]{img/octane_carbon_breakdown}
        \caption{The \ce{O_x} production budgets from octane degradation attributed to the number of carbon atoms in (a) MCM v3.2, (b) RADM2, (c) RACM and (d) RACM2.}
        \label{f:octane_carbons}
    \end{center}
\end{figure} 

The RADM2, RACM and RACM2 octane TOPP value time series presented in the paper differ from the MCM v3.2 time series by not reaching their maximum TOPP value on the second day.  
The attribution of \ce{O_x} production from octane degradation in \mbox{MCM v3.2}, RADM2, RACM and RACM2 to the number of carbon atoms of the degradation products is depicted in Figure \ref{f:octane_carbons}. 

First day \ce{O_x} production is similar between the mechanisms. 
However second day \ce{O_x} production in RADM2, RACM and RACM2 from octane degradation products having a carbon number between five and three is lower than in MCM v3.2. 
There is also no \ce{O_x} production from degradation products having six carbon atoms. 
Thus octane is broken down so quickly that it cannot reach maximum \ce{O_x} production on the second day, which is a feature of alkane degradation.
