
\begin{figure}
    \begin{center}
        \includegraphics[width=\textwidth]{img/TOL_MCM_CRI_HO2x_intermediates}
        \caption{The \ce{HO2_x} production budgets from toluene degradation attributed to the responsible reactions in (a) MCM v3.2 and (b) CRI v2.}
        \label{f:toluene_HO2x}
    \end{center}
\end{figure} 

{
    \renewcommand{\arraystretch}{1.3}
    \begin{table}
        \begin{center}\small
            \begin{tabular}{lP{6.8cm}P{3.0cm}}
                \hline \hline
                \textbf{Mechanism} & \textbf{Photolysis Pathway} & \textbf{Rate Parameter} \\ \hline \hline
                \multirow{3}{*}{MCM v3.2} & GLYOX + hv = CO + CO + H2 & J$_{31}$ \\
                & GLYOX + hv = HCHO + CO & J$_{32}$ \\
                & GLYOX + hv = CO + CO + HO2 + HO2 & J$_{33}$ \\ \hline
                CRI v2 & CARB3 + hv = CO + CO + HO2 + HO2 & J$_{33}$ \\ \hline \hline
            \end{tabular}
            \caption{Glyoxal photolysis in MCM v3.2 and CRI v2 with specified rate parameters.}
            \label{t:glyoxal}
        \end{center}
    \end{table}
}

The paper showed that the CRI v2 maximum daily toluene TOPP value is reached on the second day whilst in the MCM v3.2 this is reached on the first. 
Figure \ref{f:toluene_HO2x} illustrates the \ce{HO2_x} production budget allocated to the responsible reactions for both the MCM v3.2 and CRI v2. 
The \ce{HO2_x} production from the reaction of CARB3 and OH in CRI v2 has a larger contribution than its corresponding reaction (\mbox{GLYOX + OH}) in the MCM v3.2.

Despite glyoxal being represented as CARB3 in CRI v2 and GLYOX in MCM v3.2, there are many differences in how glyoxal chemistry is treated. 
In CRI v2, CARB3 is only produced from aromatic degradation whilst GLYOX is a degradation product of many other non-aromatic NMVOCs in MCM v3.2. 

Glyoxal degradation is through reaction with OH radical and photolysis in CRI v2. 
Extra degradation options are available in MCM v3.2. 
Moreover, the rate constant for the reaction with OH radical is $\sim$ 15\% faster in CRI v2 than in MCM v3.2. 

Glyoxal has three available photolysis pathways in MCM v3.2 and only one in \mbox{CRI v2}. 
These photolysis pathways and their rate parameters are outlined in Table \ref{t:glyoxal}. 
The additional photolysis pathways in MCM v3.2 are non-\ce{HO2_x} producing pathways leading to less \ce{HO2_x} production.

The combination of the higher rate constant for the glyoxal reaction with OH radical and additional \ce{HO2_x} production during CRI v2 photolysis are responsible for the higher \ce{HO2_x} production in CRI v2. 
