All chemical mechanisms were adapted into the modularised KPP \citep{Damian:2002} format from their original format for use in the MECCA boxmodel \citep{Sander:2005} as modified by \citep{Butler:2011}.

The MCM v3.2 \citep{Jenkin:1997, Jenkin:2003, Saunders:2003, Bloss:2005, MCM_Site} was the reference mechanism. 
Its approach to inorganic chemistry, dry deposition, photolysis and peroxy radical--peroxy radical reactions were applied to all mechanisms. 

\subsection{Photolysis} \label{ss:photolysis}
Photolysis was parameterised as a function of the solar zenith angle as per the MCM approach \citep{Saunders:2003}. 
Species from reduced mechanisms having a direct \mbox{MCM v3.2} counterpart were assigned the corresponding MCM v3.2 photolysis rate. 
Otherwise the recommended mechanism rate was used to determine the appropriate MCM v3.2 photolysis rate. 
In some cases this was the MCM v3.2 photolysis rate closest in magnitude to that specified in the mechanism. 

For example, the organic nitrate species ONIT in RACM2 has a photolysis rate of $1.96 \times 10^{-6}$ s$^{-1}$ that was compared to the MCM v3.2 organic nitrate photolysis rates \mbox{(J$_{51}$ -- J$_{57}$)}. 
The rate parameter J$_{54}$ had the most similar magnitude and was assigned as the ONIT photolysis rate in RACM2.

Photolysis of a reduced mechanism species was sometimes represented by more than one \mbox{MCM v3.2} photolysis reaction. 
The original mechanism product yields were preserved using combinations of the MCM v3.2 rates. 
This approach was used for RADM2 glyoxal photolysis. 
Glyoxal photolysis in RADM2 is described by \reactionref{r:GLY_hv_1} and \reactionref{r:GLY_hv_2}.
\begin{reactionlist}
    \reactionitem{GLY + h$\nu$}{$0.13$ HCHO + $1.87$ CO + $0.87$ \ce{H2}}{new}{r:GLY_hv_1}
    \reactionitem{GLY + h$\nu$}{$0.45$ HCHO + $1.55$ CO + $0.8$ \ce{HO2} + $0.15$ \ce{H2}}{new}{r:GLY_hv_2}
\end{reactionlist} 
Whereas in the MCM v3.2, \reactionref{r:GLYOX_hv_1}, \reactionref{r:GLYOX_hv_2} and \reactionref{r:GLYOX_hv_3} are prescribed for glyoxal photolysis with the rates J$_{31}$, J$_{32}$ and J$_{33}$ respectively.
\begin{reactionlist}
    \reactionitem{GLYOX + h$\nu$}{$2$ CO + $2$ \ce{H2}}{new}{r:GLYOX_hv_1}
    \reactionitem{GLYOX + h$\nu$}{$2$ CO + $2$ \ce{HO2}}{new}{r:GLYOX_hv_2}
    \reactionitem{GLYOX + h$\nu$}{HCHO + CO}{new}{r:GLYOX_hv_3}
\end{reactionlist}
\reactionref{r:GLY_hv_1} product yields were retained using a photolysis rate of \mbox{$0.87$ J$_{31}$ + $0.13$ J$_{32}$}, whilst for \reactionref{r:GLY_hv_2} the rate \mbox{$0.15$ J$_{31}$ + $0.45$ J$_{32}$ + $0.4$ J$_{33}$} was used.
Table \ref{t:glyoxal_photolysis} illustrates the product yield calculations.
{
    \renewcommand{\arraystretch}{1.3}
    \begin{table}
        \begin{center}\small
            \begin{tabular}{lP{4.8cm}P{7.5cm}}
                \hline \hline
                & \textbf{Rate} & \textbf{MCM v3.2 Product Yields} \\ \hline \hline
                \multirow{3}{*}{\reactionref{r:GLY_hv_1}} & $0.87$ J$_{31}$ & $1.74$ CO + $0.87$ \ce{H2} \\
                & $0.13$ J$_{32}$ & $0.13$ CO + $0.13$ HCHO \\ \cline{2-3}
                & $0.87$ J$_{31}$ + $0.13$ J$_{32}$ & $1.87$ CO + $0.13$ HCHO + $0.87$ \ce{H2} \\ \hline
                \multirow{4}{*}{\reactionref{r:GLY_hv_2}} & $0.15$ J$_{31}$ & $0.30$ CO + $0.15$ \ce{H2} \\
                & $0.45$ J$_{32}$ & $0.45$ CO + $0.45$ HCHO \\
                & $0.4$ J$_{33}$ & $0.80$ CO + $0.80$ \ce{HO2} \\ \cline{2-3}
                & $0.15$ J$_{31}$ + $0.45$ J$_{32}$ + $0.4$ J$_{33}$ & $1.55$ CO + $0.45$ HCHO + $0.80$ \ce{HO2} + $0.15$ \ce{H2} \\
                \hline \hline
            \end{tabular}
            \caption{Calculation of glyoxal MCM v3.2 photolysis rates retaining RADM2 glyoxyl photolysis product yields.}
            \label{t:glyoxal_photolysis}
        \end{center}
    \end{table}
} 

\subsection{Organic Peroxy Radical Self and Cross Reactions} \label{ss:peroxy_radical_reactions}
{
    \renewcommand{\arraystretch}{1.3}
    \begin{table}
        \begin{center}\small
            \begin{tabular}{lP{8.8cm}P{2.8cm}}
                \hline \hline
                \textbf{Mechanism} & \textbf{Reaction} & \textbf{Rate Constant} \\ \hline \hline
                \multirow{3}{*}{MCM v3.2} & C2H5O2 = C2H5O & \textit{k}*RO2*$0.6$ s$^{-1}$ \\
                & C2H5O2 = C2H5OH & \textit{k}*RO2*$0.2$ s$^{-1}$ \\
                & C2H5O2 = CH3CHO & \textit{k}*RO2*$0.2$ s$^{-1}$ \\ \hline 
                \multirow{3}{*}{MOZART-4} & C2H5O2 + CH3O2 = $0.7$ CH2O + $0.8$ CH3CHO + HO2 \newline \hspace*{3.5cm} + $0.3$ CH3OH + $0.2$ C2H5OH & $2 \times 10^{-13}$ cm$^3$ molecules$^{-1}$ s$^{-1}$ \\
                & C2H5O2 + C2H5O2 = $1.6$ CH3CHO + $1.2$ HO2 \newline \hspace*{3.5cm} + $0.4$ C2H5OH & $6.8 \times 10^{-14}$ cm$^3$ molecules$^{-1}$ s$^{-1}$\\ \hline 
                MOZART-4 & \multirow{2}{*}{C2H5O2 = $0.8$ CH3CHO + $0.6$ HO2 + $0.2$ C2H5OH} & \multirow{2}{*}{$2 \times 10^{-13}$*RO2 s$^{-1}$} \\
                modified & & \\ \hline \hline
            \end{tabular}
            \caption{Ethyl peroxy radical (\ce{C2H5O2}) self and cross organic peroxy reactions in the MCM v3.2 and MOZART-4 mechanisms including rate constants. \mbox{\textit{k} = $2$($6.6 \times 10^{-27}\exp(365/T))^{\frac{1}{2}}$ molecules$^{-1}$ s$^{-1}$} and RO2 is the sum of all organic peroxy radical mixing ratios.}
            \label{t:RO2}
        \end{center}
    \end{table}
}

{
    \renewcommand{\arraystretch}{1.3}
   \begin{table}
        \begin{center}\small
            \begin{tabular}{lP{6.8cm}P{4.8cm}}
                \hline \hline
                \textbf{Reactants} & \textbf{Products} & \textbf{Rate Constant} \\ \hline \hline
                \multirow{2}{*}{MO2 + MO2} & \multirow{2}{*}{$0.74$ HO2 + $1.37$ HCHO + $0.63$ MOH} & $9.4 \times 10^{-14}\exp{(390/T)}$ \\ & & cm$^3$ molecules$^{-1}$ s$^{-1}$ \\
                \multirow{2}{*}{MO2} & \multirow{2}{*}{$0.37$ HO2 + $0.685$ HCHO + $0.315$ MOH} & $9.4 \times 10^{-14}\exp{(390/T)}$*RO2 \\ & & s$^{-1}$ \\ \hline
                \multirow{2}{*}{ETHP + MO2} & HO2 + $0.75$ HCHO + $0.75$ ACD & $1.18 \times 10^{-13}\exp{(158/T)}$ \\ & \hspace*{5mm} + $0.25$ MOH + $0.25$ EOH & cm$^3$ molecules$^{-1}$ s$^{-1}$ \\
                \multirow{2}{*}{ETHP} & $0.63$ HO2 + $0.065$ HCHO + $0.75$ ACD  & $1.18 \times 10^{-13}\exp{(158/T)}$*RO2 \\ & \hspace*{5mm} + $0.25$ EOH & s$^{-1}$ \\ \hline \hline
            \end{tabular}
            \caption{Dermination of ETHP pseudo-unimolecular reaction and rate constant in RACM2 including rate constants. RO2 is the sum of all organic peroxy radical mixing ratios.}
            \label{t:ETHP}
        \end{center}
    \end{table}
}

Reactions of organic peroxy radicals (\ce{RO2}) with other organic peroxy radicals are divided into self (\ce{RO2 + RO2}) and cross (\ce{RO2} + R$^{\prime}$\ce{O2}) reactions. 
These reactions are typically represented in chemical mechanisms as bimolecular reactions which would cause ambiguities when implementing the tagging scheme. 
Namely, which tag to be used for the products of reactions between \ce{RO2} having different tags. 
To avoid such ambiguities, the MCM v3.2 approach to self and cross \ce{RO2} reactions is used -- each \ce{RO2} species reacts with the pool of all other \ce{RO2} at a single uniform rate. 
This was represented as a pseudo-unimolecular reaction whose rate constant includes a factor `RO2' which was the sum of the mixing ratios of all organic peroxy radicals \citep{Saunders:2003}.

The pseudo-unimolecular reaction products and their yields were determined by one of two methods.
Firstly, by using the \ce{RO2 + RO2} reaction and halving the product yields. 
This is demonstrated for the MOZART-4 treatment of the ethyl peroxy radical in \mbox{Table \ref{t:RO2}}. 
Alternatively, the \ce{RO2 + CH3O2} reaction was used and the products due to \ce{CH3O2} were removed. 
As an example, Table \ref{t:ETHP} outlines the steps taken to determine the ETHP pseudo-unimolecular reaction in RACM2. 

First the products due to MO2, representing \ce{CH3O2} in RACM2, are determined as outlined above using the \mbox{MO2 + MO2} reaction. 
The MO2 product yields are then subtracted from the \mbox{ETHP + MO2} reaction. 
Any products having a negative yield were not included in the final pseudo-unimolecular reaction.

The methyl acyl peroxy radical (\ce{CH3C(O)O2}) was the exception to the above approach. 
Although most mechanisms include a \mbox{\ce{CH3C(O)O2 + CH3C(O)O2}} reaction, its pseudo-unimolecular reaction was derived by subtracting the \ce{CH3O2} product yields from the \mbox{\ce{CH3C(O)O2 + CH3O2}}. 
This approach was used as the \mbox{\ce{CH3C(O)O2 + CH3O2}} reaction is the most significant reaction for \ce{CH3C(O)O2}.

The rate constant for each pseudo-unimolecular reaction was taken as that of the \mbox{\ce{RO2 + CH3O2}} reaction multiplied by an `RO2' factor, which is the sum of the mixing ratios of all organic peroxy radicals. 
The \mbox{\ce{RO2 + CH3O2}} rate constant was chosen as this is the most likely reaction to occur in the atmosphere. 

Model runs using the original and modified approach to the \ce{RO2}--\ce{RO2} reactions for each mechanism were performed.
The resulting \ce{O3} concentration time series were compared and shown in Figure \ref{f:O3_concentrations}.

\begin{figure}
    \begin{center}
        \includegraphics[width=\textwidth]{img/O3_mixing_ratio_comparison}
        \caption{\ce{O3} mixing ratio time series for each reduced mechanism using the original and modified approach to \ce{RO2}--\ce{RO2} reactions}
        \label{f:O3_concentrations}
    \end{center}
\end{figure}


\subsection{Dry Deposition}

Dry deposition velocities were taken from the MCM v3.2. 
The MCM v3.2 dry deposition velocities of the same chemical functional group were used for mechanism species without direct MCM v3.2 analogues. 
For example, the dry deposition velocity of PAN-like species in all mechanisms was equivalent to that of the PAN species in the MCM v3.2.

\subsection{Negative Product Yield Treatment}

Some mechanisms include reactions where products have a negative yield. 
These reactions were re-written including an operator species with a positive yield as the analysis tools used in this study do not allow negative product yields. 
The operator species acts as a sink for the original product by immediately reacting with the original product generating a `NULL' product. 

For example, in RADM2 the OH + cresol (CSL) reaction has negative OH yield \reactionref{r:CSL_OH}.
\begin{reactionlist}
    \reactionitem{CSL + OH}{0.1 \ce{HO2} + 0.9 \ce{XO2} + 0.9 \ce{TCO3} - 0.9 OH}{new}{r:CSL_OH}
\end{reactionlist}
The negative OH yield was adapted to a positive operator (OHOP) yield \reactionref{r:CSL_OHOP} which then immediately reacts with OH giving a `NULL' product with a rate constant of \mbox{$8.0 \times 10^{-11}$ cm$^3$ s$^{-1}$} \reactionref{r:OHOP_OH}. 
Thus preserving the OH yields using the original mechanism.
\begin{reactionlist}
    \reactionitem{CSL + OH}{0.1 \ce{HO2} + 0.9 \ce{XO2} + 0.9 \ce{TCO3} + 0.9 OHOP}{new}{r:CSL_OHOP}
    \reactionitem{OHOP + OH}{NULL}{new}{r:OHOP_OH}
\end{reactionlist}
