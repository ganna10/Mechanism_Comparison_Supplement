
The emitted NMVOC are typical of Los Angeles as described in \citet{Baker:2008}.
The MCM v3.2, v3.1 \citep{Jenkin:1997, Saunders:2003, Jenkin:2003} and \mbox{CRI v2} \citep{Jenkin:2008} explicitly represent all of these NMVOC.

NMVOC representation in all other mechanisms required mapping to specific mechanism species.
This mapping followed the recommendations on the literature of the mechanism.
Table \ref{t:VOC_mapping} describes the mechanism species used for mapping the initial NMVOC.
Table 2 of the main article shows the final mapping of each NMVOC to each mechanism species.

{
    \renewcommand{\arraystretch}{1.3}
    \begin{sidewaystable}
        \begin{center}\footnotesize
            \begin{tabular}{llP{8cm}|llP{8cm}}
                \hline \hline
                \textbf{Mechanism} & \textbf{Species} & \textbf{Description} & \textbf{Mechanism} & \textbf{Species} & \textbf{Description} \\
                \hline \hline
                \multirow{8}{2cm}{MOZART-4 \citep{Emmons:2010}} & C2H6 & Ethane & \multirow{13}{2cm}{RACM2 \citep{Goliff:2013}} & ETH & Ethane \\
                & C3H8 & Propane & & HC3 & OH rate constant (298 K, 1 atm) less than $3.4 \times 10^{-12}$ cm$^3$ s$^{-1}$ \\
                & BIGALK & Lumped alkanes \mbox{C \textgreater $3$} & & HC5 & OH rate constant (298 K, 1 atm) between $3.4 \times 10^{-12}$ and $6.8 \times 10^{-12}$ cm$^3$ s$^{-1}$ \\
                & C2H4 & Ethene & & HC8 & OH rate constant (298 K, 1 atm) greater than $6.8 \times 10^{-12}$ cm$^3$ s$^{-1}$ \\
                & C3H6 & Propene & & ET{E} & Ethene \\
                & BIGENE & Lumped alkenes \mbox{C \textgreater $3$} & & OLT & Terminal alkenes \\
                & ISOP & Isoprene & & OLI & Internal alkenes \\
                & TOLUENE & Lumped aromatics & & ISO & Isoprene \\ \cline{1-3}
                \multirow{10}{2cm}{RADM2 \citep{Stockwell:1990}} & ETH & Ethane & & BEN & Benzene \\
                & HC3 & OH rate constant (298, 1 atm) between $2.7 \times 10^{-13}$ and $3.4 \times 10^{-12}$ & & TOL & Toluene and less reactive aromatics \\
                & HC5 & OH rate constant (298, 1 atm) between $3.4 \times 10^{-12}$ and $6.8 \times 10^{-12}$ & & XYM & m-Xylene \\
                & HC8 & OH rate constant (298, 1 atm) greater than $6.8 \times 10^{-12}$ & & XYO & o-Xylene \\
                & OL2 & Ethene & & XYP & p-Xylene \\ \cline{4-6}
                & OLT & Terminal Alkenes & \multirow{8}{2cm}{CBM-IV \citep{Gery:1989}} & PAR & Paraffin carbon bond \ce{C-C} \\
                & OLI & Internal Alkenes & & ETH & Ethene \\
                & ISO & Isoprene & & OLE & Olefinic carbon bond \ce{C=C} \\
                & TOL & Toluene and less reactive aromatics & & ALD2 & High molecular weight aldehydes \\
                & XYL & Xylene and more reactive aromatics & & ISOP & Isoprene \\ \cline {1-3}
                \multirow{10}{2cm}{RACM \citep{Stockwell:1997}} & ETH & Ethane & & TOL & Toluene \\
                & HC3 & OH rate constant (298 K, 1 atm) less than $3.4 \times 10^{-12}$ cm$^3$ s$^{-1}$ & & XYL & Xylene \\ 
                & HC5 & OH rate constant (298 K, 1 atm) between $3.4 \times 10^{-12}$ and $6.8 \times 10^{-12}$ cm$^3$ s$^{-1}$ & & FORM & Formaldehyde \\ \cline{4-6}
                & HC8 & OH rate constant (298 K, 1 atm) greater than $6.8 \times 10^{-12}$ & \multirow{8}{2cm}{CB05 \citep{Yarwood:2005}} & ETHA & Ethane \\ 
                & ET{E} & Ethene & & PAR & Paraffin carbon bond \ce{C-C} \\
                & OLT & Terminal alkenes & & OLE & Terminal olefin carbon bond \ce{R-C=C} \\
                & OLI & Internal alkenes & & FORM & Formaldehyde \\
                & ISO & Isoprene & & ISOP & Isoprene \\
                & TOL & Toluene and less reactive aromatics & & TOL & Toluene and other monoalkyl aromatics \\
                & XYL & Xylene and more reactive aromatics & & XYL & Xylene and other polyalkyl aromatics \\
                \hline \hline
            \end{tabular}
            \caption{Description of primary mechanism species used for mapping emitted NMVOCs.}
            \label{t:VOC_mapping}
        \end{center}
    \end{sidewaystable}
}
