%
Figure $4$ of the main article shows the first day TOPP values of the VOC obtained in each reduced mechanism compared to the MCM v3.2.
The first day TOPP values of $2$-methylpropene in RACM, RACM2, MOZART-4, CBM-IV and CB05 signify differences in its degradation to the MCM v3.2.

The variation between RACM, RACM2 and MCM v3.2 arises from differences in the ozonolysis rate constant of $2$-methylpropene.
This rate constant is an order of magnitude faster in RACM and RACM2 than in MCM v3.2 as the RACM, RACM2 rate constant is a weighted mean of the ozonolysis rate constants of each VOC represented as OLI \citep{Stockwell:1997, Goliff:2013}.
The faster rate constant promotes increased radical production leading to more \ce{O_x} in RACM and RACM2 than the MCM v3.2.

$2$-methylpropene is represented as BIGENE in MOZART-4. 
The degradation of BIGENE produces \ce{CH3CHO} through the reaction between NO and the $2$-methylpropene peroxy radical, whereas no \ce{CH3CHO} is produced during $2$-methylpropene degradation in the MCM v3.2.
\ce{CH3CHO} \mbox{initiates a} degradation chain producing \ce{O_x} involving \ce{CH3CO3} and \ce{CH3O2} leading to more \ce{O_x} in MOZART-4 than \mbox{MCM v3.2}.

CBM-IV and CB05 represent $2$-methylpropene as a combination of aldehydes and PAR, the \ce{C-C} bond \citep{Gery:1989, Yarwood:2005}.
This representation of $2$-methylpropene does not produce the $2$-methylpropene peroxy radical, whose reaction with NO is the main source of \ce{O_x} production in all other mechanisms.

